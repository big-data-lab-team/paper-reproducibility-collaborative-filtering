\documentclass[10pt, conference, compsocconf]{IEEEtran}

% packages
\usepackage{algorithm}
\usepackage{algorithmic}
\usepackage{amsfonts} % for R symbol (the set of real numbers)
\usepackage{color}
\usepackage[pdftex]{graphicx}
\usepackage[bookmarks=false]{hyperref}
\hypersetup{colorlinks=true,linkcolor=black,citecolor=black,filecolor=black,urlcolor=blue}
\usepackage{mathtools}
\usepackage{multirow}
\usepackage{stmaryrd} % for llbracket and rrbracket
\usepackage{subcaption}
\DeclarePairedDelimiter{\ceil}{\lceil}{\rceil}
\DeclarePairedDelimiter{\floor}{\lfloor}{\rfloor}

% new commands
\newcommand{\todo}[1]{\marginpar{\parbox{18mm}{\flushleft\tiny\color{red}\textbf{TODO}:
      #1}}}

\newcommand{\note}[1]{
  \color{blue}\emph{[Note: #1]}
  \color{black}
}

\begin{document}

\title{Predicting computational reproducibility of scientific pipelines using collaborative filtering}

\author{Soudabeh Barghi, Lalet Scaria, Tristan Glatard\\
  Department of Computer Science and Software Engineering\\ Concordia University, Montreal, Quebec, Canada\\
  {first.last}@concordia.ca\\
  $^*$ These authors have contributed equally
}

\maketitle

\begin{abstract}
\end{abstract}


\section{Introduction}

% Computational reproducibility is an issue, for instance among
% different operating systems (refer to Glatard FINF 2015,
% Gronenschild 2012).

% Scientific data analysis pipeline executions are long (give examples
% from neuroimaging).

% Refer to Germain et al 2008.

% Define subjects.

% Problem: predict the reproducibility of pipeline files from other
% subjects and the first files produced by a pipeline. Restrict the
% study to binary classification.


\section{Method}

% What is specific to our problem compared to regular collaborative filtering:
%+ subjects are equivalent to users. Not all subjects have the same input data. Anatomical variability, acquisition variability (e.g., 1 subject may have multiple T1s).
%+ files are produced in a specific order (movies aren't),
% which add constraints on the training set (cannot sample late files and not early ones).
%+ utility matrix is not sparse: we can potentially populate it completely. Which means that we can decide precisely which samples we need (active sampling). Therefore we can take the first row and first column to avoid cold start issues.

\subsection{Collaborative filtering using ALS}

% Summarize Koren, Bell and Volinsky: https://dl.acm.org/citation.cfm?id=1608614

% Explain that you have binary classes (rounding)

% Biases

\subsection{Sampling the Training Set}

% Explain all sampling techniques (maybe rename 4, 5, and 6)
% 1. random unreal (baseline)
% 2. columns
% 3. rows
% 4. random real
% 5. random diagonal
% 6. random triangular
% For every method, show a training matrix to illustrate for a given ratio. 

\section{Dataset}

% Describe your dataset: pipeline(s) used, input data, operating systems (CentOS5, 6, 7), matrix.

% 1. Prefreesurfer (what you have)
% 2. Freesurfer, with the same subjects as in Prefreesurfer.
% 3. PostFreesurfer
% 4. fmriVolume

\section{Results}

% Present your results: accuracy, ROC curves, transparency matrix, factors. 

% Try with different numbers of factors. 

\section{Discussion}

% Not all subjects behave the same, which motivates the Big Data approach. 

% Which sampling method is best

% Interpreting the factors? Factors reflect the pipeline definition. 

% How can this be used in practice

% What are the limitations

\section{Conclusion}

% Summary of the results and discussion. The take-home message.

\section*{Acknowledgment}

\bibliographystyle{IEEEtran}
\bibliography{IEEEabrv,biblio.bib}


\end{document}
